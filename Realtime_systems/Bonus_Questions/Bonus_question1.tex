\documentclass[../../../main.tex]{subfiles}
\graphicspath{../../../Figures/}

\title{Bonus questions}
\author{Fredrik Mårtensson}
\date{\today}

\begin{document}
\maketitle
%\tableofcontents
\section{Bonus questions 1}
\subsection{What are Cyber-Physical Systems and in what sense are they different from Embedded Systems?}
A cyber-physical system is a system that works between different physical and software systems through wired or wireless communication.  A cyber-physical system control the physical processes and with help of feedback it can affect the computation on how a code is executed in order to optimise, simplify and improve a process. CPS is widely used in systems that depends on communication and feedback.

Example on CPS include areas were properties such as intervention, precision, coordination and efficiency is needed. Products that is included for this is robotic surgery, autonomic vehicles and air traffic control. Both autonomic vehicles and air traffic control work through wired and wireless communication. This is to ensure that two planes or cars keep their distance and talk to each other in order not to collide. They both need to know were the other are and other data such as speed, acceleration maybe if one car is trying to change lane.

Difference between CPS and a embedded system is that it act as the main core were the code executes. Embedded systems is included in a CPS but is only one part of the connected system. Example on a embedded system could be a system that does not take the environment into account, for example if the airbag in a car only checked one sensor instead of communicating with all other systems in a car. One theory is that in order for a system to be defined as CPS, there is a need for two or more embedded systems that have a linked feedback that goes both ways. This could be that a system sends a request and a system receives it and confirms the request through feedback.



\end{document}