\documentclass[../../../main.tex]{subfiles}
\graphicspath{../Figures/}
\begin{document}
\section{Summery of papers}
\subsection{Challenges of Securing IIoT}
\begin{enumerate}
    \item Basically: The network is not safe, security is an illusion.
    \item Man in the middle is dangerous
    \item Using same keys for networks is bad and should be replaced with strong passwords that is used on only one device instead of hole network
\end{enumerate}
\subsubsection{Conclusion}
This  article explored  the  challenges  of  securing  metrol-ogy  data  for  the  IIoT,  where  we  investigated  seven  areasin  particular.  Namely  the  challenges  in:  IIoT  system  modelsecurity, IIoT end device security, IIoT network security, IIoTcloud  security,  IIoT  application  security,  IIoT  trust,  and  IIoTattacks.  In  response  to  these,  we  have  highlighted  some  ofthe  outstanding  problems,  the  issues  when  creating  real-lifeimplementations,  and  the  research  needed  to  solve  this  for  afuture  IIoT.  As  mentioned  earlier,  nowadays  there  is  a  de-mand on custom security solutions: Rather than using genericsolutions,  security  experts  are  devising  highly  customizedsecurity  solutions  for  each  network  that  is  being  designed.This brings the advantages of rapid act on fixing the securityvulnerabilities  of  that  specific  network  by  releasing  patchestimely manner and/or enhancing the security level in the nextrelease  by  closing  all  the  gaps  that  are  recognized.  In  thiscontext, IIoT systems security is projected to follow this trendof customized approach in ensuring the security of IIoT valuechain. Hence, this brief summary of security measures alongwith presented topics and ideas will help researchers not onlyenhancing  security-awareness  in  IIoT  as  a  whole  system  butalso in securing its sub-components such as devices, networks,clouds,  and  applications.  In  these  areas  there  is  much  futurework  left  to  be  performed,  which  is  why  our  own  researchwill primarily be focused on the following items:
\begin{enumerate}
    \item Security requirement analysis of IIoT (obtaining vulner-abilities list according to the various attack vectors)
    \item Design  of  a  customized  security  architecture  for  a  con-ceptual IIoT setup
    \item Theoretical and practical security analysis of the proposedsolution (customized security architecture)
    \item Comparison of the proposed security solution to its’ rivalsin the literature (if any).
\end{enumerate}
\subsection{CoAP: An Application Protocol for Billions of Tiny Internet Nodes}
\begin{enumerate}
    \item Constrained Application Protocol (CoAP)
    \item HTTP designed to work in proxies, not very resourse friendly. Therefor we need REST, implemented in CoAP that is working with limited resourses "Compressed HTML"
    \item CoAP: Less complex than HTTP, UDP, message layer that handles retransmission, GET,PUT,POST, DELETE included
    \item REST is related to reverse proxy that creates a gateway on the client
    \item Sending data as blocks instead of IP fragmentations
    \item GET=Expensive in HTTP, CoAP:  asynchronous, client acting as observer if server accept it
    \item Instead of SSL/TSL, CoAP: on top of  Datagram  Transport  Layer  Security  
\end{enumerate}
\subsubsection{Conclusion}
As  a  tiny  but  well-designed  and  quite  functional  stand-in  for  HTTP,  CoAP  is  slated  to  become  a  ubiquitous  application  protocol  for  the  future  Internet  of  Things  —  or,  really,  the  “Internet  of  Innumerable  Embedded Systems."

\subsection{Formal security analysis of LoRaWAN}
\begin{enumerate}
    \item LoRa is a proprietary physical layer protocol that facilitate slow-power and long-distance communication up to 20 Km by using Chirp Spread Spectrum(CSS) modulation technique.
    \item Over-The-Air Activation (OTAA), Activation By Personalization (ABP)
    \item Challanges: Cryptographic primitives, Key preloading, Infrastructure trust, roaming
\end{enumerate}
\subsection{Generic SoC Architecture Components}
\begin{enumerate}
    \item Bluetooth systems operate in the 2.4GHz ISM band and use the frequency hopping spread spectrum (FHSS) method instead of DSSS to spread their signals.
    \item Maxiumum: 1600 hops per second
    \item version 2.0 - 3 Mbps
    \item Attack tools against bluetooth: Ubertooth
    \item radio waves to transmit data
    \item Hacking: Bluejacking, Bluesnarfing
    \item Protocol: physical layer
\end{enumerate}
\subsection{Industrial Internet of Things: Challenges,Opportunities, and Directions}
\begin{enumerate}
    \item 
\end{enumerate}
\end{document}
